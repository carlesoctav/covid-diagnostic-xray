%-----------------------------------------------------------------------------%
\chapter{PENUTUPAN}

\section{Kesimpulan}
Pada makalah ini, kami menggunakan model CNN sederhana untuk mengkklasifikasikan pasien positif COVID-19 dan negatif COVID-19 dengan menggunakan hasil rontgen pasien. Model ini hanya menggunakan 4 layer konvolusi dan max-pooling, serta 1 \textit{fully-connected layer}. Model mendapatkan hasil yang baik pada semua metrik standar selama proses \textit{training, validation}, dan \textit{testing}, dengan begitu,  model ini tidaklah \textit{overfitting}. Model bisa mendapatkan hasil akurasi sebesar 94\% pada data test, oleh  karena itu model tersebut sudah cukup baik dalam melakukan klasifikasi positif
dan negatif COVID-19 dari hasil rontgen paru-paru.

\section{Saran}
Saran yang bisa kami sampaikan adalah diperlukannya komparasi model dengan model lainnya. Selain itu, diperlukan juga pengujian ulang model terhadap dataset (rontgen paru-paru) yang terbuka lainnya.



