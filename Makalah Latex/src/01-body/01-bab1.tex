%-----------------------------------------------------------------------------%
\chapter{PENDAHULUAN}
%-----------------------------------------------------------------------------%
\section{Latar Belakang}
Sejak akhir tahun 2019, dunia terlanda penyakit dari jenis \textit{coronavirus} baru, yaitu COVID-19. COVID-19 adalah penyakit yang dapat menyebabkan gangguan pernapasan, nyeri dada, dan bahkan kematian. Karena penyebaran virus ini sangat cepat, WHO mengklasifikasikan penyakit ini sebagai pandemi. Akibatnya, berbagai pihak dalam dunia kesehatan tengah mencari cara-cara untuk menanganinya. Salah satu komponen yang penting dalam penanganan wabah COVID-19 adalah diagnosis. Karena dampaknya pada paru-paru, umumnya orang yang terjangkit virus COVID-19 akan menunjukkan kelainan pada citra paru-parunya. Metode citra medis paru-paru yang sering digunakan adalah radiografi rontgen. Artinya, seorang radiolog dapat mendeteksi kelainan dan menggunakan hasil rontgen untuk membantu dalam diagnosis COVID-19. Namun, menimbang jumlah kasus yang besar dan kemiripan hasil rontgen pasien pneumonia dari COVID-19 dan virus atau bakteri lainnya, pendeteksian berbasis komputer akan sangat membantu radiolog dalam mencapai sebuah diagnosis. Diperlukan sebuah sistem yang dapat mengklasifikasi citra rontgen dan menerka ada atau tidaknya peran COVID-19 di situ. Atas dasar tujuan ini, kelompok kami telah membangun sebuah model berbasis \textit{Convolutional Neural Network (CNN)} untuk melaksanakan klasifikasi biner dari citra rontgen.

\section{Tujuan}
Tujuan proyek ini adalah untuk mengimplementasikan CNN pada masalah klasifikasi gambar. Dari hasil rontgen, kami bertujuan untuk membuat model yang dapat memprediksi terjangkit COVID-19 atau tidaknya pasien tersebut dengan akurasi besar.