%
% Halaman Abstrak
%
% @author  Andreas Febrian
% @version 1.00\section{Abstrak}


%

\chapter*{Abstrak}

\vspace*{0.2cm}
{
Pandemi penyakit COVID-19 telah menimbulkan perlunya peningkatan teknik telaah medis dalam hal mengidentifikasi dan membantu dokter mendiagnosis penyakit ini. Umumnya, pasien COVID-19 menunjukkan berbagai gejala, antara lain demam, batuk, dan kelelahan. Sebab gejala ini juga muncul pada pasien pneumonia, hal ini menimbulkan komplikasi dalam deteksi COVID-19, terutama selama musim flu. Studi awal menunjukkan bahwa kelainan pada gambar rontgen dada pasien yang terinfeksi COVID-19 dapat digunakan untuk membantu diagnosis penyakit. Teknik \textit{Convolutional Neural Network} (CNN), pada khususnya, umum digunakan untuk masalah klasifikasi gambar. Dalam proyek akhir ini, kami menggunakan CNN untuk melakukan klasifikasi gambar rontgen pasien COVID-19. Kami menggunakan 13.808 gambar yang terdiri dari hasil rontgen 10.192 pasien yang negatif COVID-19 dan 3.616 pasien yang positif COVID-19. Hasil eksperimen telah menunjukkan akurasi keseluruhan setinggi ... 


	\bigskip

	Kata kunci: Klasifikasi, COVID-19, CNN
}

\newpage