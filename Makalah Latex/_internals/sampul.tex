%
% Sampul Laporan

%
% @author  unknown
% @version 1.01
% @edit by Andreas Febrian
%

\begin{titlepage}
    \begin{center}    
        \begin{figure}
            \begin{center}
                \includegraphics[width=2.5cm]{pics/makara.png}
            \end{center}
        \end{figure}    
        \vspace*{0cm}
        \textbf{
        	UNIVERSITAS INDONESIA\\
        }
        
        \vspace*{1.0cm}
        % judul thesis harus dalam 14pt Times New Roman
        \textbf{KLASIFIKASI KASUS POSITIF DAN NEGATIF COVID-19 DARI HASIL RONTGEN PARU-PARU DENGAN CONVOLUTIONAL NEURAL NETWORK} \\[1.0cm]

        \vspace*{2.5 cm}    
        % harus dalam 14pt Times New Roman
        \textbf{TUGAS MAKALAH}

        \vspace*{3 cm}       
        % penulis dan npm
        \textbf{CARLES OCTAVIANUS (20068568613)}\\
        \textbf{AGUSTINUS BRAVY TETUKO OMPUSUNNGGU (2006521300)} \\
        \textbf{
        RENDI HARTADI (2006533383)}\\

        \vspace*{5.0cm}

        % informasi mengenai fakultas dan program studi
        \textbf{
        	FAKULTAS MATEMATIKA DAN ILMU PENGETAHUAN ALAM
        \\	PROGRAM STUDI MATEMATIKA \\
        	DEPOK \\
        	JUNI 2022
        }
    \end{center}
\end{titlepage}
